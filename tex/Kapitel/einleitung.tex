%Internet impasse
Ein Konzept dem Internet Impasse mit flexibler Architektur \underline{[WAS IST DAS?]} und Handhabbarkeit zu begegnen, wurde in der Netzwerkvirtualisierung (NV) gefunden. \cite{anderson2005overcoming, bays2012security, fischer2013virtual} Sie basiert auf Knoten- (z.B. Xen) und Linkvirtualisierung \underline{[ANGABE ZU BEIDEM?]} und erlaubt so von der tatsächlichen physischen Hardware und Netzinfrastruktur (NI) beinahe unabhängige logische bzw. virtuelle Netzwerke einzurichten, welche nach außen hin den Anschein physischer Netzwerke erwecken. Die Möglichkeit mehrere virtuelle Maschinen (VMs) pro physischem Host und verschiedene heterogene virtuelle Netzwerke (VNs) auf demselben physischen Substratnetz zu betreiben befördert die Flexibilität der Netzwerkarchitektur und wirkt dem \underline{Internet Ossification Problem} \cite{anderson2005overcoming} entgegen.

%Vorteile von NV
Die großen Vorteile der NV liegen in der Abstraktion von der eingesetzten Hardware. Das Erstellen, Verändern, Migrieren, Zurücksetzen und Löschen von Maschinen funktioniert genauso einfach wie der Umgang mit Dateien, was eine dynamische Nutzung des Netzwerkes erlaubt. Virtuelle Maschinen und Netzwerke eigenen sich auch als Testumgebung. Einerseits werden bestehende Systeme im Fehlerfall nicht direkt beeinträchtigt. Andererseits kann neuer Code leicht in verschiedenen Umgebungen (Windows, Linux, verschiedengroßer RAM, mit oder ohne Software-Developer-Kits etc.) ohne zusätzliche Hardware getestet und später einfach ausgerollt werden.

NV eröffnet eine Unterteilung des klassischen Internetserviceproviders (ISP) in Service-Provider (SP) und Infrastructure-Provider (InP). Damit gewonnene Freiheiten durch z.B. jeweils unabhängige Technologieentscheidungen sind besonders für Unternehmen interessant, die die Hardwarebasis ihrer Dienste nicht selbst unterhalten wollen.  \cite{wang2016towards}\\
Das Anbieten von Software und Hardware als on-demand Ressourcen wird wegen des geringen Wartungsaufwand, verminderter Hardwarekosten durch Koexistenz mehrerer Mieter, aber v.a. wegen Automatisierbarkeit in der Programmierung der Netzwerkumgebung vereinfacht. Dass InPs nicht mehr streng durch Hardware limitiert sind, begünstigt Skalierbarkeit und bspw. lastbedingte Migration von VMs auf andere physische Hosts.\\
Auch für den Kunden bietet NV Vorteile: Unternehmen bezahlen nur noch für diejenigen Ressourcen, die gerade in Anspruch genommen werden. Hochqualitative Hardware kann so zu einem Bruchteil ihres Preises erworben und ungenutzte Hardware reduziert werden. Durch dynamisches Skalieren (z.B. in Zeiten hoher Last) kann die eigene IT-Landschaft mühelos vergrößert werden.

%Ziele von NV
Die durch NV gewonnene Flexibilität und Kostenreduzierung begünstigen z.B. auch das Outsourcing von Rechenleistung, Speicher, Inhalten und Netzwerk. Dadurch wird Soft- und Hardware einfacher nutzbar gemacht und Geschäftsprozesse befördert. 
Die damit einhergehende Verantwortungsübertragung erfordert eine Anpassung des Risikomanagements und IT-Sicherheitstechnische Arbeiten zur Erhaltung der klassischen C.I.A.-Aspekte. Wegen der gemeinsam genutzten Hardware kommt aus Sicht des Kunden besonders der Isolation und dem Datenschutz eine wichtige Rolle zu.

% 1. Wie kann ich VN deployen?
Essentielle Komponente der NV ist die Wahl der Zuordnung von virtuellen zu physischen Knoten und Links, das Virtual Network Embedding (VNE), welches auf Basis verschiedener Kriterien geschehen kann. Das theoretische Problem des VNE wurde bislang hauptsächlich unter Performanceaspekten optimiert und Sicherheitsbelange dabei weitgehend außer Acht gelassen.\\
% 2. Wie kann ich es "sicher" deployen?
Bekannte Sicherheitsmechanismen wie Verschlüsselung, Firewalls, Intrusion Detection Systeme etc. können zwar auf den virtuellen Komponenten des Netzwerks implementiert werden. Die Sicherheit von Nutzerdaten lässt sich dadurch aber wegen der heterogenen und stark dynamischen Struktur virtueller Umgebungen jedoch nicht garantieren. Zusätzlich gehen Vorteile der Netzwerkvirtualisierung durch den zusätzlichen Overhead verloren. \cite{gong2016virtual}\\
% 4. VNE + Sec-Aspekte
Eine mögliche Lösung hierzu ist das Integrieren von Sicherheitsaspekten bereits in den VNE-Prozess, was eine der größten Herausforderungen in der Netzwerkvirtualisierung darstellt. \cite{fischer2013virtual}
Werden virtuelle Netzwerke entsprechend ihrer Sicherheitsanforderungen bereits auf Substratknoten mit hinreichender Schutzfunktion wie beispielsweise Firewall abgebildet, so kann Overhead durch zusätzliche Sicherheitstechnik im laufenden Betrieb des VNs reduziert werden. Nicht allen Problemen bzw. Sicherheitsrisiken lässt sich allerdings auf diese Weise begegnen. \cite{bays2012security, gong2016virtual, wang2016towards} 


Diese Arbeit klassifiziert Gefahren im Kontext virtualisierter Netzwerke und analysiert zwei unterschiedliche Ansätze zur Schaffung eines möglichst hohen Sicherheitsniveaus bereits zum Zeitpunkt des VNE-Prozesses.\\
Dazu wird zuerst das VNE-Problem in Kapitel \ref{sec:VNE-Problem} dargestellt. Kapitel \ref{sec:gefahren} untersucht Sicherheitsanforderungen an virtuelle Netzwerkstrukturen und klassifiziert Sicherheitsrisiken, die sich in deren Kontext ergeben. Zwei Möglichkeiten zur Vermeidung von Gefahren, denen bereits im VNE-Prozess begegnet werden kann, werden im Kapitel \ref{sec:svne} betrachtet. Nach einer Diskussion in dieser Arbeit offengebliebener Probleme in Kapitel \ref{sec:schluss} wird mit einem Ausblick abgeschlossen.