\documentclass{lni}


\IfFileExists{latin1.sty}{\usepackage{latin1}}{\usepackage{isolatin1}}
\usepackage{graphicx}
\usepackage{url}
\usepackage{listings}
\usepackage{nameref}
\graphicspath{{pix/}{}}

\author{Gerhard Gr?schl, Miran Mizani\\\{gerhard.groeschl, miran.mizani\}@campus.lmu.de\\\\
Seminar: Trends in Mobilen und Verteilten Systemen \\Wintersemester 2016/2017\\\\
Lehrstuhl f?r Mobile und Verteilte Systeme\\Institut f?r Informatik\\Ludwig-Maximilians-Universit? M?nchen\\\\
Betreuer: Michael T. Beck\\\textit{eingereicht am \today}}
\title{Sicherheitsaspekte bei Deployment virtueller Netzwerkinfrastrukturen}



\begin{document}
\maketitle

\vfill
FEHLT [GG]: Optimierung der Grafiken, Erkl?ung der Ansatz2-Algorithmen, Ansatz2 Komplexit? und Performanz, Vergleich, Einf?bung der Gleichung Ansatz1, Legende maMo Ansatz2

FRAGEN[GG]: mathematische Modelle rausnehmen?

\begin{abstract}
Durch die ??rst effiziente Nutzung von Hardware mittels Virtualisierung steigt die Nachfrage nach virtualisierten Infrastrukturen enorm. Serviceprovider m?ssen die Hardwarebasis f?r ihre Dienste nicht mehr selbst unterhalten und geben ihre dahingehende Verantwortung an Infrastrukturanbieter weiter. Um die Sicherheit dieser Strukturen nicht zu vernachl?sigen, arbeiten viele Forscher in diesem Bereich und versuchen effiziente Algorithmen mit integrierter Beachtung der Sicherheitsaspekte zu finden. Diese Arbeit soll einen ?erblick und eine Klassifizierung der Gefahren, die solche Konstrukte betreffen, sowie eine Analyse zweier unterschiedlicher Ans?ze zur Vermeidung m?glichst vieler Risiken zum Zeitpunkt der Planung ?bermitteln.
%Hier folgt eine kurze Zusammenfassung des Themas sowie der wichtigsten Erkenntnisse und Ergebnisse. L?ge: ca. 200 W?rter.
\end{abstract}


\newpage
\tableofcontents
\newpage

\section{Einleitung}
\label{sec:Einleitung}
%Internet impasse
Ein Konzept dem Internet Impasse mit flexibler Architektur \underline{[WAS IST DAS?]} und Handhabbarkeit zu begegnen, wurde in der Netzwerkvirtualisierung (NV) gefunden. \cite{anderson2005overcoming, bays2012security, fischer2013virtual} Sie basiert auf Knoten- (z.B. Xen) und Linkvirtualisierung \underline{[ANGABE ZU BEIDEM?]} und erlaubt so von der tatsächlichen physischen Hardware und Netzinfrastruktur (NI) beinahe unabhängige logische bzw. virtuelle Netzwerke einzurichten, welche nach außen hin den Anschein physischer Netzwerke erwecken. Die Möglichkeit mehrere virtuelle Maschinen (VMs) pro physischem Host und verschiedene heterogene virtuelle Netzwerke (VNs) auf demselben physischen Substratnetz zu betreiben befördert die Flexibilität der Netzwerkarchitektur und wirkt dem Internet Ossification Problem \cite{anderson2005overcoming} entgegen.

%Vorteile von NV
Die großen Vorteile der NV liegen in der Abstraktion von der eingesetzten Hardware. Das Erstellen, Verändern, Migrieren, Zurücksetzen und Löschen von Maschinen funktioniert genauso einfach wie der Umgang mit Dateien, was eine dynamischere Nutzung des Netzwerkes erlaubt. Virtuelle Maschinen und Netzwerke eigenen sich auch als Testumgebung. Einerseits werden bestehende Systeme im Fehlerfall nicht direkt beeinträchtigt. Andererseits kann neuer Code nun leicht in verschiedenen Umgebungen (Windows, Linux, verschiedengroßer RAM, mit oder ohne Software-Developer-Kits etc.) ohne zusätzliche Hardware getestet und später einfach ausgerollt werden.

NV eröffnet eine Unterteilung des klassischen Internetserviceproviders (ISP) in Service-Provider (SP) und Infrastructure-Provider (InP). Damit gewonnene Freiheiten durch z.B. jeweils unabhängige Technologieentscheidungen[wang2016towards] sind wohl besonders für Unternehmen interessant, die die Hardwarebasis ihrer Dienste nicht mehr selbst unterhalten wollen.\\
Das Anbieten von Software und Hardware als on-demand Ressourcen wird wegen geringeren Wartungsaufwand, verminderte Hardwarekosten durch Koexistenz mehrerer Mieter, aber v.a. wegen Automatisierbarkeit in der Programmierung der Netzwerkumgebung vereinfacht. Dass InPs nicht mehr streng durch Hardware limitiert sind, begünstigt Skalierbarkeit und bspw. lastbedingte Migration von VMs auf andere physische Hosts.\\
Auch für den Kunden bietet NV Vorteile: Unternehmen bezahlen nur noch für diejenigen Ressourcen, die gerade in Anspruch genommen werden. Hochqualitative Hardware kann so zu einem Bruchteil ihres Preises erworben und ungenutzte Hardware reduziert werden. Durch dynamisches Skalieren (z.B. in Zeiten hoher Last) kann die eigene IT-Landschaft mühelos vergrößert werden.

%Ziele von NV
Das Outsourcing von Rechenleistung, Speicher, Inhalten und Netzwerk soll Soft- und Hardware einfacher nutzbar machen und Geschäftsprozesse befördern. Die damit einhergehende Verantwortungsübertragung erfordert eine Anpassung des Risikomanagements und IT-Sicherheitstechnische Arbeiten zur Erhaltung der klassischen C.I.A.-Aspekte. Wegen der gemeinsam genutzten Hardware kommt aus Sicht des Kunden besonders der Isolation und dem Datenschutz eine wichtige Rolle zu.\\
Bekannte Sicherheitsmechanismen wie Verschlüsselung, Firewalls, Intrusion Detection Systeme etc. können zwar auf den virtuellen Komponenten des Netzwerks implementiert werden. Die Sicherheit von Nutzerdaten könne dadurch aber wegen der heterogenen und stark dynamischen Struktur virtueller Umgebungen jedoch nicht garantiert werden. Ferner dürften Vorteile der Netzwerkvirtualisierung durch den zusätzlichen Overhead verloren gehen. \cite{gong2016virtual}\\
Eine mögliche Lösung hierzu ist das Integrieren von Sicherheitsaspekten bereits in die Zuordnung von virtuellen zu physischen Knoten und Links. Dieser Virtual Network Embedding (VNE) Prozess stellt eine der größten Herausforderungen in der Netzwerkvirtualisierung dar. \cite{fischer2013virtual}
Werden virtuelle Netzwerke entsprechend ihrer Sicherheitsanforderungen bereits auf Substratknoten mit hinreichender Schutzfunktion wie beispielsweise Firewall abgebildet, so kann Overhead durch zusätzliche Sicherheitstechnik im laufenden Betrieb des VNs reduziert werden.

[TODO]
Derzeit wird versucht \cite{bays2012security, gong2016virtual, wang2016towards} derartige Probleme bereits im VNE-Algorithmus anzugehen [SVNE Security-aware VNE|security by design]. Nicht allen Problemen bzw. Sicherheitsrisiken kann aber auf diese Weise begegnet werden. %(= Überleitung zur Gliederung und zweiten Kapitel?)


Diese Arbeit soll Gefahren im Kontext virtualisierter Netzwerke klassifizieren und zwei unterschiedliche Ansätze zur Schaffung eines möglichst hohen Sicherheitsniveaus bereits zum Zeitpunkt des VNE-Prozesses analysiert.\\
Dazu wird zuerst das VNE-Problem in Kapitel \ref{sec:VNE-Problem} \textit{\nameref{sec:VNE-Problem}} dargestellt. Kapitel \ref{sec:gefahren} \textit{\nameref{sec:gefahren}} untersucht Sicherheitsanforderungen an virtuelle Netzwerkstrukturen und klassifiziert Sicherheitsrisiken, die sich in deren Kontext ergeben. Zwei Möglichkeiten zur Vermeidung von Gefahren, denen bereits im VNE-Prozess begegnet werden kann, werden im Kapitel \ref{sec:svne} \textit{\nameref{sec:svne}} betrachtet. Nach einer Diskussion in dieser Arbeit offengebliebener Probleme in Kapitel \ref{sec:offenefragen} \textit{\nameref{sec:offenefragen}} wird mit einem Ausblick abgeschlossen.
[KAPITELNAMEN-REFERENZIERUNG NÖTIG? LIEBER OHNE?]



% Die Ausarbeitung zum Seminar soll dem Layout der \textit{GI-Edition Lecture Notes in Informatics (LNI)} entsprechen. Die verwendete Literatur wird in der beiliegenden Datei \verb+literatur.bib+ verwaltet. Eine Referenz kann mittels des \verb+\cite{}+--Kommandos eingef?gt werden, z.B.

% \subsection{Hinweise f?r Abbildungen}
% Abbildungen m?ssen als \texttt{.pdf}, \texttt{.png}, oder \texttt{.jpg} eingebunden werden. Beispiel:
% \begin{figure}[htb]
%   \begin{center}
%     \includegraphics[width=1cm]{gilogo.pdf}
%     \caption{\label{logo}Das Logo der GI}
%   \end{center}
% \end{figure}


\section{Das Virtuel Network Embedding Problem}
\label{sec:VNE-Problem}
Um den Anforderungen der heutigen Gefahren zu gen?gen, ist es unumg?glich s?tliche Grunds?ze der IT-Sicherheit m?glichst fr?h in die Planung der gew?nschten Infrastruktur miteinzubeziehen. Nicht nur, weil das nachtr?liche Schlie?n von Sicherheitsl?cken und Hinzuf?gen von Sicherheitskomponenten in finanzieller und zeitlicher Hinsicht zehnmal so teuer ist wie die initiale Beachtung dieser Aspekte, sondern weil die vollst?dige Sicherheit eines nachger?steten Systems kaum gew?rleistet werden kann \cite{Cole}."`Security-by-Design"' ist einer der wichtigsten Begriffe bei der Planung ?ffentlich zug?glicher Netzwerkstrukturen. Dementsprechend gro?ist die Nachfrage nach VNE-Algorithmen, die bereits beim Prozess des Mappings m?glichst viele Sicherheitsaspekte beachten und abdecken. Die vorausgehende Klassifizierung der bekannten Gefahren in "`VNE-relevant"' und "`nicht-VNE-relevant"' bildet die Grundlage f?r unsere weiteren Untersuchungen. Um zuvor noch einen genaueren Einblick in die Problematik von VNE zu gew?ren, beginnen wir zun?hst mit der Erl?terung des VNE-Problems sowie einem ?erblick ?ber die verschiedenen erfolgreichen Strategien bestehender VNE-Algorithmen, welche die Sicherheitsaspekte noch nicht beachten. 

Betrachtet man das zugrundeliegende physische System als einen ungerichteten Graphen
\begin{center}
	\includegraphics[width=0.7\textwidth]{physical_structure2.pdf}\newline
	  % \caption{\label{logo}(A bis E benennt die Knoten, L1 bis L6 benennt die Links, die Zahlen darunter stehen f?r die jeweiligeverf?gbare Resourcenmenge(abstrakt))}
\end{center}
\vspace*{1cm}

%Grafik
und extrahiert die vorhandenen Elemente sowie deren Werte, erh?t man die Basismenge
\begin{center}
G\textsuperscript{S} = \{N\textsuperscript{S}, L\textsuperscript{S}\}\newline

(N steht f?r die Menge der physischen Knoten, L f?r die Menge der physischen Links)
\end{center}

\begin{center}
G\textsuperscript{S} = 
\{\{
(A\textsuperscript{S},50),(B\textsuperscript{S},30),(C\textsuperscript{S},70),
(D\textsuperscript{S},40),(E\textsuperscript{S},20)\},
\newline\{
(L1\textsuperscript{S},3),(L2\textsuperscript{S},9),(L3\textsuperscript{S},8),
(L4\textsuperscript{S},10),(L5\textsuperscript{S},4),(L6\textsuperscript{S},7)\}\}
\end{center}
in welche die virtuellen Strukturen eingebettet werden sollen.

Ein "`virtual network request"' (in Folge "`VNR"' genannt) kann hierbei ebenfalls als ein ungerichteter Graph gesehen
\begin{center}
	\includegraphics[width=0.4\textwidth]{VNR1.pdf}\newline
	  % \caption{\label{logo}(A bis E benennt die Knoten, L1 bis L6 benennt die Links, die Zahlen darunter stehen f?r die jeweiligeverf?gbare Resourcenmenge(abstrakt))}
\end{center}
%Grafik
und genauso in eine Menge ?bersetzt werden.\newline
\begin{center}
G\textsuperscript{V} = \{N\textsuperscript{V}, L\textsuperscript{V}\}\newline
(N steht f?r die Menge der virtuellen Knoten, L f?r die Menge der virtuellen Links)
\end{center}

\begin{center}
G\textsuperscript{V} = \{\{
(A\textsuperscript{V},20),(B\textsuperscript{V},10),(C\textsuperscript{V},25)\} , \{
(L1\textsuperscript{V},5),(L2\textsuperscript{V},3),(L3\textsuperscript{V},4)\}\}
\end{center}

\begin{center}
Gesucht ist nun eine Abbildungsfunktion f : G\textsuperscript{V} $\rightarrow$ G\textsuperscript{S}
\end{center}

In der Regel ist es ?blich, nicht nur eine einzelne VNRs auf ein physisches System abzubilden, sondern gleich mehrere auf einmal. Ebenso w?e es theoretisch m?glich, dass ein Infrastruktur-Provider gleich mehrere getrennte unabh?gige physische Strukturen betreibt, und somit mittels VNE das beste System f?r die Abbildung der Menge von VNRs eroiren m?chte. Dies k?nnte beispielsweise der Fall sein, wenn ein Serviceprovider ein VNR mit der Bedingung "`alle Elemente m?gen sich am selben Standort befinden, egal an welchem"' in Auftrag gibt, und der Infrastruktur-Provider ?ber mehrere Hardware-Standorte verf?gt.
\newline
% Gv=...
Die Attributmenge bei den grundlegenden VNE-Algorithmen beschr?kt  sich zumeist auf Rechnerleistung und Bandbreite. 
Lokalit?, GPU-Leistung und RAM-Menge w?en einige weitere m?gliche Standard-Attribute. Ein gro?s Problem bei der Berechnung des optimalen Mappings findet sich bei der Berechnungszeit. Die endliche Beschr?kung der Knoten- sowie Link-Resourcen und die "`on-line nature"' von VNRs stellen zus?zliche Hindernisse dar. Da selbst L?sungen f?r einfache Anfragen (geringe Anzahl von Knoten und Links), welche wenige Attribute beinhalten, exponentiellen Rechenaufwand ben?tigen, steigt der Aufwand sowohl mit der Anzahl der abzubildenden Knoten und Links, als auch mit steigender Attributanzahl dementsprechend. Teilweise gilt das Problem als rechnerisch unl?sbar, grunds?zlich aber befinden wir uns im Komplexit?sbereich der NP-Vollst?digkeit \cite{SVNE2}. Da die Erh?hung der Attributmenge - wie bereits genannt - grunds?zlich nicht positiv zur Laufzeit der existierenden Algorithmen beitr?t, werden wir den Komplexit?sbereich beim Hinzuf?gen von Sicherheitsanforderungen nicht verlassen. Dennoch gibt es M?glichkeiten, die den Rechenaufwand reduzieren k?nnen. Im Folgenden widmen wir uns allerdings zuerst einer ?ersicht ?ber die Gefahren, welche bei VNE eine Rolle spielen.



\section{Sicherheitsaspekte virtueller Netzwerk-Infrastrukturen / -systeme}
\label{sec:gefahren}

\subsection{Sicherheitsanforderungen}
\label{subsec:gefahren_anforderungen}
[FALLS NÖTIG / GEWÜNSCHT. SONST WIRD DIESES KAPITEL NICHT AUSGEARBEITET UND ENTFÄLLT] Netzwerkvirtualisierung ist eine Methode unabhängige und voneinander isolierte logische Netzwerke auf einem gemeinsamen physischen Netzwerk zu betreiben. Dieses Kapitel widmet sich möglichen Anforderungen, die an ein so solches System gestellt werden könnten.


\begin{itemize}
\item	Anonymität und Privatsphäre von Nutzern (vertrauenswürdige Paketweiterleitung, wenn das virt. Netzwerk bei einem Drittanbieter gehostet ist)
\item	VM Interkonnektivität
\item	Möglichkeit zur Zusammenarbeit mit nicht-virtualisierten Netzen und Anwendungen\\	Bsp. Günstige Webhosting-Lösungen auf vServern sollen auch von nicht-virt. Maschinen aus erreichbar sein.
\item	Angriffe von Endnutzern sollten nicht in der Lage sein das Substratnetz zu beeinträchtigen, während Attackenverkehr hingegen identifiziert und gefiltert werden können soll.
\item	Substratnetz sollte nicht VN-Verkehr monitoren, sniffen, manipulieren, stören/unterbrechen (können).
\item	Umgekehrt sollte auch das VN keine Informationen über das Substratnetz sammeln (können).
\item	Substratnetz sollte effektive Kontrollmechanismen gegen unerlaubte Erlangung von Infos aus/über das VN bereitstellen.
\item	\dots
\end{itemize}


\paragraph*{Mögliche Richtlinien}

\begin{itemize}
\item ~[IN DIESER ARBEIT WERDEN FOLGENDE BEGRIFFE IN DIESEM SINNE VERSTANDEN / KURZE BEGRIFFSDEFINITION]
\item Isolation ist Hauptaufgabe / Kernpunkt [-> Ausbauen]	
	\begin{itemize}
	\item Keine gegenseitige Beeinflussung der VMs oder VNs
	\item Keine Beeinflussung zwischen VMs verschiedener Mieter / VNs.
	\item Von zwei VMs gemeinsam genutzte Ressourcen sind logisch voneinander getrennt. Kein ungewollter direkter Informationsfluss.[Fußnote: Ausführliche Betrachtung indirekten Informationsflusses z.B. via verdeckter Kanäle oder side channel Attacken würde den Umfang dieser Arbeit übersteigen.]
	\item Problem: Kunde weiß nicht, ob die „Nachbar-VM“ vertrauenswürdig ist.
	\item Die gehosteten VNs sollten keine Attacken gegen privilegierte Informationen der Substratinfrastruktur fahren.
	\end{itemize}
\item	Datenschutz
\item	Vertraulichkeit, Integrität, Verfügbarkeit (CIA)
	\begin{itemize}
	\item Vertraulichkeit: Zwei VMs innerhalb desselben VNs sollen in der Lage sein frei miteinander zu kommunizieren, ohne dann dabei ein Dritter durch Belauschung an Kommunikationsinhalte gelangen kann.
	\item Integrität: Ein virt. Knoten / VM soll nicht in der Lage sein, (Kommunikations-)Daten Anderer zu manipulieren und die eigene Urheberschaft der Veränderungen einer anderen Instanz zuzuschreiben. 
	\end{itemize}
\item	Informationsflusskontrolle
\item	Mitglieder- und Zulassungsverwaltung in Virt. Netzen.
\end{itemize}

\paragraph*{Herausforderungen:} %[natarajansecurity]
\begin{itemize}
	\item Kann das Substratnetz Paketvermittlung effektiv durchführen, ohne Informationen über das VN zu sammeln oder Privatsphäre zu verletzen?
	\item \dots
\end{itemize}

\subsection{Herk?mmliche Gefahren in Netzinfrastrukturen}
\label{subsec:gefahren_allg}



\subsection{Neue Verwundbarkeiten in virtualisierten Umgebungen}
\label{subsec:gefahren_virt}
Sicherheitsrisiken in virtualisierten Netzinfrastrukturen lassen sich auf verschiedene Weisen wie z.B. nach ISO/OSI-Schicht, nach Verletzung der klassischen C.I.A.-Aspekte, nach Schicht in der NV-Architektur, oder aus Sicht des SPs bzw. InPs klassifizieren. Da dieses Kapitel sich aber auf durch Netzvirtualisierung gegenüber herkömmlichen Netzinfrastrukturen neu hinzukommende Risiken konzentriert, liegt der folgenden Klassifizierung zur Verdeutlichung der Angriffsrichtungen die in Abbildung \ref{fig:gefahren_klassifizierung} dargestellte Struktur zugrunde, die sich auf der Drei-Schichtenarchitektur der Netzwerkvirtualisierung (Abbildung \ref{fig:gefahren_dreiEbenenDerVirtualisierung}) ableitet.

Sicherheitsrisiken werden zunächst nach solchen technischer, organisatorischer bzw. unternehmerischer und rechtlicher Art geordnet. Im Zentrum der Betrachtung stehen dabei die technischen Risiken, welche wiederum nach Angriffsrichtungen ‚\textit{von NI ausgehend}‘, ‚\textit{von VN/VM ausgehend}‘ und ‚\textit{vom User ausgehend}‘ gegliedert werden. In jeder dieser Kategorien wird nach Angriffsziel ‚\textit{gegen NI}‘, ‚\textit{gegen VN/VM}‘ und ‚\textit{gegen User}‘ unterteilt. \\
Da die Kategorien ‚\textit{I.1 von NI ausgehend gegen NI}‘ und ‚\textit{III.3 vom User ausgehend gegen User}‘ in virtualisierten Netzwerkumgebungen keine neuen Angriffsszenarien eröffnen, werden sie hier dem Ziel des Kapitels entsprechend nicht tiefergehend ausgeführt. 


\begin{figure}[htb]
	\begin{center}
	\includegraphics[width=\textwidth]{gefahren_klassifizierung.pdf}
	\caption{\label{fig:gefahren_klassifizierung} Klassifizierung von Gefahren in virtualisierten Netzwerken
		\newline NI = Netzwerkinfrastruktur / Substratnetz. Der physische Host einer VM ist Teil der NI.
		\newline User = Endsystem bzw. in VNs implementierter Service oder Nutzer
		\newline Kategorie I.1 und III.3 beinhalten keine durch NV neu hinzukommenden Risiken.}
	\end{center}
\end{figure}





\subsubsection{Technischer Art}
\label{subsubsec:gefahren_virt_technisch}
Dieses Kapitel analysiert sich durch NV ergebende Sicherheitsrisiken technischer Art.
Der Kapitelaufbau folgt der Klassifizierung in Abbildung \ref{fig:gefahren_klassifizierung}. 


\paragraph{I.2 \& I.3 Von NI ausgehend gegen VN/VM und User.}
\label{parag:vonNI}
Physische Hosts bieten ihren VMs Ressourcen an. Alle Dienste und Anwendungen der VMs werden letztlich auf dem physischen Host ausgeführt und auch alle Daten auf ihm gespeichert. Dies eröffnet für den physischen Host prinzipiell die Möglichkeit eines Monitorings der VM-Aktivitäten, was ab einer gewissen Intensität über Verwaltungsbelange hinausgehen und Privats"-phären"-anforderungen widersprechen wird. Auf demselben Weg lassen sich auch vertraulichkeitsverletzende Sniffing- oder Spoofing-Attacken gegen VM bzw. VN starten. \\
Da alle ihre Rechenoperationen letztlich auf dem physischen Host ausgeführt werden, ist es für eine VM nur schwer möglich, sich gegen solche Angriffe zu wehren. Eine Verschlüsselung der eigenen Daten kann einer VM nicht ohne Weiteres helfen, da auch die Schlüsseldatei innerhalb der VM und damit auf dem physischen Host gespeichert werden muss.\\
Auch Manipulation des legitimen Datenverkehrs, (gezieltes) Verwerfen von empfangenen Paketen bzw. das Einschleusen schadhafter Nachrichten bieten eine weitere Möglichkeit der Kompromittierung, gegen die VNs und Endsysteme wohl schutzlos ausgeliefert sind.
%\underline{[AUSFÜHRLICHER?]}\\
Durch derartige Aktionen oder auch unzureichende Sicherungsmaßnahmen gegen Datenabfluss kann der physische Host in den SLAs vereinbarte Bestimmungen verletzen, was gerade bei Drittanbietern als Hostingpartner eine Rolle spielt.




\paragraph{Von VN/VM ausgehend}
\label{parag:vonVN}
Die gemeinsame Nutzung von Ressourcen wie Speicher und Netzwerkkarten eröffnet VMs neue Angriffsmöglichkeiten gegen ihr Substratnetz, gegen andere VNs/VMs und gegen User.


\begin{itemize}
\item \textbf{II.1 Von VN/VM ausgehend gegen NI/ihren physischen Host.~~}
Das Bereitstellen von Ressourcen für VMs ist auch für den physischen Host nicht ohne Risiko. Schadhafte oder bösartige VMs können Verwundbarkeiten ihres physischen Host über zugeteilte Ressourcen angreifen. Ohne hinreichende Restriktionen kann eine VM über ihr zugeteiltes Kontingent hinaus bspw. wichtige Speicherbereiche manipulieren oder z.B. durch übermäßige Reservierung von CPU-Zeiten eine Denail of Service Attacke gegen den physischen Host bzw. das Substratnetz fahren. Da Host und virtuelle Netztopologie aus der Ferne konfigurierbar sind, stellt das Einschleusen von konstruierten Nachrichten des verwendeten Netzwerkmanagementprotokolls auf oder durch die Netzwerkkarte des physischen Hosts einen weiteren gefährlichen Angriffsvektor dar.\\
Nach Eindringen in oder Übernahme des Hosts -- einem „break of isolation“ im ersten Sinne \cite{wu2010network} -- könnte eine schadhafte VM dann ihr Kontingent an Ressourcen beeinflussen, netztopologische Informationen sammeln, andere Netzwerkressourcen oder -infrastrukturkomponenten angreifen und so beispielsweise Dienste anderer VMs oder VNs behindern.

\item \textbf{II.2 Von VN/VM ausgehend gegen VN/VM.~~}
Neben den herkömmlichen Angriffsszenarien zwischen Maschinen im selben Netzwerk, ergeben sich v. a. aus der gemeinsamen Nutzung von Ressourcen neue Angriffsvarianten.\footnote{Da die Angriffstechniken von VMs gegen VMs auf anderen physischen Hosts vergleichbar mit herkömmlichen Angriffen in nicht-virtualisierten Netzinfrastrukturen sein dürften, werden hier hauptsächlich VMs auf demselben physischen Host betrachtet.} 
Ein Angreifer kann sich nun gezielt Ressourcen auf denjenigen physischen Maschinen mieten, auf denen auch sein Angriffsziel gehostet wird, um so erleichterten Zugang zu deren Verwundbarkeiten zu erlangen. Durch Eindringen oder Übernehmen gewisser Ressourcen des gemeinsamen physischen Hosts kann eine schadhafte VM ggfs. Verwundbarkeiten anderer VMs ausnutzen oder durch Cross-VN-side-channel-Attacken vertrauliche Informationen gewinnen und Daten manipulieren. Ein Beispiel einer Integritätsverletzung mittels einer solchen Attacke findet sich in \cite{ristenpart2009hey}. \\
Da i.d.R. nur virtuelle Netzwerkkarten zugeteilt werden, kann jede VM potentiell den gesamten Datenverkehr aller VMs bzw. VNs auf derselben physischen Netzwerkkarte lesen. Ein Monitoring anderer VMs auch aus anderen VNs, ein „break of Isolation“ im zweiten Sinne bedroht deren Vertraulichkeit. Durch Belauschen des Netzwerkverkehrs lassen sich ggfs. auch Dienste gemeinsam gehosteter VN reproduzieren und dadurch bspw. ein live Videostreaming breiter zugänglich machen. \cite{natarajansecurity}\\
Sollte es einer VM gelingen kritische Teile ihres physischen Hosts zu übernehmen, so stehen ihr zusätzlich die im Abschnitt \textit{\nameref{parag:vonNI}} aufgeführten Angriffsvektoren offen.


\item \textbf{II.3 Von VN/VM ausgehend gegen User.~~}
Auch ein virtuelles Netzwerk kann mit herkömmlichen Methoden wie Monitoring der Nuteraktivitäten oder dem Einschleusen von konstruierten Nachrichten zur Störung oder Abbruch von Peer-to-Peer-Verbindungen Einfluss auf den Nutzverkehr seiner User nehmen.
\end{itemize}



\paragraph{Vom User ausgehend}
\label{parag:vonUser}
Einem Benutzer oder schadhaften Anwendungsprogramm stehen auch in virtualisierten Netzwerkumgebungen die bekannten Methoden der Störung durch z.B. Herbeiführen von Überlastsituationen in VN oder NI offen.

\begin{itemize}
	\item \textbf{III.1 Vom User ausgehend gegen NI.~~}
	Da sich die virtuelle Netztopologie im VNE-Prozess laufend ändert, müssen Netzwerkkomponenten wie Switches und Router dynamisch umprogrammierbar sein. Dies jedoch ermächtigt Angreifer solche ggfs. mit Codeexpliots wie Bufferoverflows o. Ä. zu kompromittieren und für ihre Zwecke zu nutzen oder einen Denail of Service herbeizuführen.\\
	Daneben besteht die Chance auch Netzwerkknoten anzugreifen. Gelingt es z.B. mit einem Rootkit wie BluePill \cite{rutkowska2008bluepilling} -- als Vorbereitung für weitere Angriffe -- einen Hypervisor zu übernehmen, wird so gleichzeitig die Kontrolle über alle gehosteten VMs erlangt. Auch eine VM lässt sich als Rootkit instrumentalisieren. \cite{wu2010network}
	\item \textbf{III.2 Vom User ausgehend gegen VN/VM.~~}
	Aus der dynamischen Natur virtueller Netzwerktopologien ergeben sich neue Verwundbarkeiten: Während der Migration im Livebetrieb eines VNs ist eine Man-in-the-Middle-Attacke möglich, mit der Informationen über und Inhalte des migrierenden VNs erlangt werden können. \cite{natarajansecurity} Auch die Manipulation von Speicherbereichen der VMs ist während der Migration umsetzbar und lässt sich sogar automatisieren. \cite{oberheide2008empirical}\\
	Die Notwendigkeit die gesamte virtuelle Netzwerkstruktur aus der Ferne umkonfigurieren zu können, erschließt weitere Angriffsziele: Attacken auf die VN-Ma"-nagement"-tools durch z.B. Cross-Site-Scripting, SQL-Injection etc. werden lohnend, da auf diese Weise effizient Kontrolle über das gesamte Netzwerk gewonnen werden kann.	
\end{itemize}



Durch Virtualisierung eröffnen sich also eine Reihe neuer Verwundbarkeiten und Angriffsmöglichkeiten technischer Art, die hauptsächlich in der gemeinsamen Nutzung von Ressourcen begründet liegen.
%Hinzu kommen Risiken ...



\subsubsection{Organisatorischer Art}
\label{subsubsec:gefahren_virt_organisatorisch}
Unter ‚organisatorischen‘ Risiken werden hier Risiken für Unternehmen verstanden, die im Zusammenhang mit Virtualisierung stehen.

Wie im Kapitel \ref{subsubsec:gefahren_virt_technisch} dargestellt, eröffnet Netzvirtualisierung eine Reihe neuer Verwundbarkeiten für gehostete Systeme. Gerade für Unternehmen stellt die teils deutliche Gefähr"-dung der Vertraulichkeit und Integrität von Firmen- oder Kundendaten ein ernstzunehmendes Problem dar. Da Netzvirtualisierung oftmals via Cloudcomputing abgewickelt wird, erhöht sich das Risiko eines Datenlecks nochmals durch den Up- und Downloadprozess von Daten.\\
Mögliche Richtlinien zur Beschränkung von Anzahl, Art oder Eigentümer gemeinsam gehosteter VNs/VMs, welche sich z.B. als komplexe Sicherheitsvektoren mit dem in Kapitel \ref{subsec:ansatz2} vorgestellten Algorithmus realisieren ließen, schränken Flexibilität der Netzarchitektur und Kostenvorteil der NV zwar ein, können jedoch bei ausreichendem Sicherheitslevel der physischen Hosts einen gewissen Schutz gewährleisten. Vor- und Nachteile eines solchen Ansatzes werden in Kapitel \ref{subsec:ansatz2} und \ref{subsec:svne_vergleich} diskutiert.

Ein großer Vorteil virtueller Maschinen besteht in der Leichtigkeit Momentaufnahmen (Snapshots) für Backup- oder Migrationszwecke zu erstellen. Beim Wiedereinspielen dieser werden jedoch ggfs. zwischenzeitlich deaktivierte Accounts, veraltete Sicherheitsrichtlinien oder mittlerweile gepatchte Schwachstellen wieder produktiv gesetzt. Organisationen müssen hierfür einen geeigneten Prozess etablieren, der auch das Patchmanagement häufig offline gehender VMs regelt. Der Umgang mit solchen VMs ist schwierig, da Würmer meist relativ schnell alle verwundbaren Systeme infizieren. Geht die VM danach offline wird Malware darin ggfs. nicht entdeckt und die Wurminfektionselle startet erneut beim Onlinegehen der VM. \cite{garfinkel2005virtual} 



\subsubsection{Rechtlicher Art}
\label{subsubsec:gefahren_virt_rechtlich}
Da Substratnetze nicht zwingend räumlich eng beschränkt sein müssen, könnte es passieren, dass gewisse Teile des virtuellen Netzes (dynamisch / lastbedingt) auf Knoten oder in ein Land gemappt werden, welches mit den Auflagen des Unternehmens zu Datenschutz, Privatsphäre oder IT-Securitystandards nicht vereinbar ist. 
Hinzu kommen eventuelle aus technischen Risiken wie Sniffing resultierende Konflikte mit rechtlichen Vorschriften zu bspw. Datenschutz.



\subsection{VNE-Relevante Gefahren}
\label{subsec:gefahren_vnerelevant}	


\section{Vermeidung von Gefahren via Secure VNE (SVNE)}
\label{sec:svne}
Da wir nun einen ?erblick zu VNE sowie den bestehenden Gefahren gegeben haben, widmen wir uns nun zwei verschiedenen SVNE-L?sungsans?zen.

\subsection{Ansatz 1}

Hierbei besch?tigen wir uns mit dem L?sungsansatz aus \cite{wang2016towards}. 
Das Hauptaugenmerk bez?glich der Sicherheitsaspekte legen Wang et al auf Traffic-Verschl?sselung und die Separierung von VMs unterschiedlicher Trust-Levels. Mittels"`Security plan design"'werden die drei folgenden strukturellen Aspekte betrachtet und in dementsprechende Levels eingeteilt.
\begin{itemize}
\item The Network plan:\newline
Ein VNR wird als Netzwerk betrachtet und seinem Level entsprechend isoliert. "`High"'fordert und beansprucht ein gesamtes Netz bzw. Subnetz der physischen Infrastruktur f?r sich. Hiermit soll die Wahrscheinlichkeit f?r DOS-Angriffe oder ?nliche vermindert werden, da"`Mehr-Parteien-Netzwerke"'die Hauptschwachstellen f?r solche Angriffe darstellen \cite{DOS}. Auch Sniffing durch kompromittierte Hosts im selben Netz wird durch diese Ma?ahme verhindert. W?rend "high"'Resourcenteilung somit komplett verweigert, l?st der Level"`medium"'zumindest ein gemeinsam genutztes Netz f?r VNRs vom selben Eigent?mer zu.

\item The Node plan:\newline
Die einzelnen virtuellen Knoten eines VNR stellen Isolierungsanforderungen an die physischen Knoten. Wie auch beim"`network plan", werden die Levels"`high","`medium"'und"`none"'definiert und umgesetzt."`high"'fordert die alleinige Existenz eines VNR-Knotens,"`medium"'l?st VNR-Knoten vom selben Eigent?mer auf ein und demselben physischen Knoten zu. Durch diesen Plan sollen Angriffe von VM zu VM ?ber gemeinsam genutzte Resourcen unterbunden werden. Zus?zlich wird das Risiko eines Angriffs vom physischen Host verringert, da der Angriffsvektor"`VM zu physischem Host"'eliminiert wird.

\item The Link plan:\newline
End-to-End(E2E), Point-to-Point(P2P) und"`none"'sind die hier w?lbaren Levels. W?rend E2E nur an den Endpunkten Verschl?sselungskapazit?en zu Verf?gung stellen muss, ben?tigt P2P diese Kapazit?en an allen Hops des abgebildeten Links. Die heutzutage sehr g?gigen man-in-the-middle Angriffe sollen dadurch wesentlich erschwert werden. 
\end{itemize}
VNRs werden nun, wie in Kapitel 2 bereits gezeigt, in ungerichtete Graphen mit Standardanforderungentransformiert. Zus?zlich werden hier auch noch Sicherheitsanforderungen integriert.\newline
\begin{center}
	\includegraphics[width=0.8\textwidth]{SVNR.pdf}\newline
\end{center}
Nun wird ein vier-stufiges Pre-Processing durchgef?hrt, um die Berechnung der optimalen Abbildung der VNRs zu vereinfachen. Hierbei werden als ersten die Standardanforderungen der virtuellen Knoten mit den zu Verf?gung stehenden Kapazit?en der physischen Knoten verglichen. Sollten physische Knoten bestimmte Anforderungen nicht erf?llen, werden sie aus der Berechung ausgeschlossen. Der zweite Schritt ist dem"`Network plan"'gewidmet. Sollte ein physischer Knoten, welcher den Anforderungen des"`Network plan"'nicht gen?gt sich in einem Kandidaten-Netzwerk befinden, wird das Subnetz aus den Berechnungen ausgeschlossen. Im dritten Schritt werden die Sicherheitsanforderungen der einzelnen Knoten betrachtet und nicht entsprechende physischen Knoten werden abermals entfernt. Der letzte Schritt widmet sich den Sicherheitsanforderungen aus dem"`Link plan"'und streicht Links aus den weiteren Berrechnungen, welche den Verschl?sselungsanforderungen der virtuellen Links nicht gen?ge tun.\newline
\begin{center}
	\includegraphics[width=1\textwidth]{pre-processing.pdf}\newline
\end{center}
Mit den aus dem Pre-Processing gewonnen Information bildet man nun einen Hilsgraphen, welcher die ?brig gebliebenen M?glichkeiten der Einbettung zeigt.\newline
\begin{center}
	\includegraphics[width=0.6\textwidth]{aux.pdf}\newline 
\end{center}
Dieses Pre-Processing reduziert das SVNE-Problem auf ein"`multi-commodity-flow"'Problem [mit einem Wert (Rohstoff?) pro Link]. \cite{MCF}
Im weiteren Vorgehen werden nun zwei F?le unterschieden."`Path-splitting", im weiteren SVNE-PS und"`no-path-splitting", im weitern SVNE-NPS, sind zus?zliche Sicherheitsvorgaben, welche im Vorfeld definiert werden m?ssen, um die Wahl des Algorithmus zu erm?glichen. Sollte SVNE-PS gew?lt werden, beziehen sich die Gleichungen nur auf die Erf?llung der geforderten Attribute.
Sollte SVNE-NPS gew?lt werden, werden die f?r Link-Abbildungen verantwortlichen Gleichungen ersetzt.\newline
\begin{center}
	\includegraphics[width=1\textwidth]{algo.pdf}\newline
\end{center}
\begin{center}
	\includegraphics[width=1\textwidth]{notations.pdf}\newline
	\newline
	\includegraphics[width=1\textwidth]{variables.pdf}\newline
\end{center}

Fehlt: Kurze Beschreibung der Punkte 1-13\newline

Die Berechnungen wurden mittels induktiver logischer Programmierung unter Verwendung von CPLEX\cite{CPLEX} und durch k-shortest-path anhand aller abzubildenden Elemente limitiert.
F?r die Testumgebungen wurden zuf?lig erzeugte VNRs mit 2 bis 10 Knoten und halbsovielen Links erzeugt. Auch die Rechenkapazit? wie Bandbreite wurden durch Zufallsgeneratoren mit Werten im Bereich von 1 bis 10 gew?lt und verteilt. Die Anzahl der VNRs betr?t einer Poisson-Verteilung nach einen Durchschnitt von 4 VNR pro 100 Zeiteinheiten, mit einer jeweiligen durchschnittlichen Einsatzzeit von 1000 Zeiteinheiten.

Die zugrundeliegenden physischen Netze wurden ebenfalls zuf?lig erzeugt und beinhalteten 10 bis 50 Knoten, sowie halbsoviele Links. Die Rechen- sowie Bandbreiten-Kapazit?en wurden gleichm?ig verteilt und enthielten Werte zwischen 1 und 50. Die Verschl?sselungskapazit?en der Knoten  wurde ?ber die gesamte Testreihe ebenfalls gleichm?ig verteilt.

Die folgende Statisk zeigt einen Durchschnittsvergleich zwischen dem, als Standard gew?lten, VNE-Algorithmus\cite{Std} und den beiden SVNE-Varianten PS und NPS. Dazu sei gesagt dass s?tliche Algorithmen ihre Berechnungen anhand des Hilfsgraphen durchgef?hrt haben und durch k=3 bzw. k=4 limitiert wurden.\newline
\begin{center}
	\includegraphics[width=0.8\textwidth]{statistic.pdf}\newline
\end{center}
Man sieht  hier einen deutlichen Zeitvorsprung von PS und NPS gegen?ber dem Standard-Algorithmus. Da hier jedoch auch der Standard-Algorithmus vom Pre-Processing profitiert, w?e ein Vergleich, welcher die Dauer des Pre-Processings mit aufnimmt und den Standard-Algorithmus ohne Pre-Processing arbeiten lie?, anschaulicher und aussagekr?tiger.
Dennoch ist ein Punkt durchaus ?berraschend und bemerkenswert: Der Standard-Algorithmus ben?tigt deutlich l?ger, obwohl er keine Sicherheitsaspekte mitbeurteilt, im Gegenzug zu seinen Kontrahenten.

In welchem Komplexit?sbereich sich das Pre-Processing befindet wird hier leider nicht erw?nt.


\subsection{Ansatz 2}

Im zweiten Ansatz widmen wir uns dem Modell aus \cite{algo2}. Dieser Ansatz arbeitet mit abstrakten Sicherheitslevels. Im Folgenden werden zwar Skalare hierf?r verwendet, was allerdings nicht zwingend so vorgesehen ist. Stattdessen w?e eine Verwendung komplexerer Sicherheitsvektoren m?glich, welche bei weitem detailliertere Sicherheitsmerkmale  beschreiben k?nnten. Des weiteren verf?gt jeder physische wie auch virtuelle Knoten ?ber zwei Sicherheitswerte: Anforderungslevel und (eigenes) Sicherheitslevel. Der Anforderungslevel definiert das Minimum-Level des Gegen?bers, der Sicherheitslevel definiert die eigenen gew?nschten Sicherheitsmerkmale. Virtuelle Links verf?gen ?ber Anforderungslevels, physische Links nur ?ber Sicherheitslevels. Die vorausgesetzten Basisanforderungen, welchen alle VNRs unterliegen, beschr?ken sich auf 4 Regeln, welche mittels der zuvor genannten Merkmale umgesetzt werden:
%\begin{itemize}%%%%%%%%%%%%%%%%%%%%%%%%%%%%%%%%%%%%%%%%%%%%%%%%%%%%%%%%%%%%%%%%%%%%%%%%%%%%%%%%%%%%%
\begin{itemize}%\begin{listing} %%% ich glaube, wenn dann {lstlisting}
\item Ein physischer Knoten sollte einen Sicherheitslevel garantieren, 
   der h?her ist als die Anforderungen der darauf abzubildenden
   virtuellen Knoten.

\item Der Sicherheitslevel des virtuellen Knotens sollte h?her sein, 
   als das Anforderungslevel des physichen Knotens.

\item Alle virtuellen Knoten, welche auf den selben physischen Knoten
   abgebildet werden, sollten ?ber einen ausreichenden Sicherheitslevel
   verf?gen.

\item Der Anforderungslevel des virtuellen Links sollte stets niedriger
   sein, als das Sicherheitslevel des physischen Links.
\end{itemize}%\end{listing}

\begin{center}
	\includegraphics[width=0.8\textwidth]{algo2graph.pdf}\newline
\end{center}
%\end{itemize}%%%%%%%%%%%%%%%%%%%%%%%%%%%%%%%%%%%%%%%%%%%%%%%%%%%%%%%%%%%%%%%%%%%%%%%%%%%%%%%%%%%%%
Wie man in Abbildung X sieht, werden auch bei diesem Modell VNRs und physische Netze in ungerichtete Graphen transformiert. Die in Klammern gestellten Parameter beschreiben Anforderungslevel(D) und Sicherheitslevel(L). Die hier durchgef?hrte Abbildung ber?cksichtigt nicht nur die Sicherheitsanforderungen aller Parteien unter Ber?cksichtigung der Basisanforderungen, sondern achtet zus?zlich noch auf Kostenminimierung. Insgesamt existieren drei Abbildungsm?glichkeiten, die gew?lte ist jedoch die g?nstigste, unter dem Aspekt keine Sicherheitsresourcen zu verschwenden. Dieser VNR h?te auch auf EDGH abgebildet werden k?nnen, wobei der Link ED ?ber einen Sicherheitslevel verf?gt, welcher h?her als n?tig ist. W?rde dies nicht beachtet werden, k?nnte es zu unn?tigen Engp?sen bei der Behandlung von VNRs mit h?heren Anforderungslevels kommen. Um ein solches Verhalten zu erreichen, wurden zus?zlich Kosten- und Nutzen-Funktionen in die Berechnung integriert.
\begin{center}
	\includegraphics[width=0.8\textwidth]{algo2.pdf}\newline
\end{center}

Fehlt: Kurze Beschreibung der Punkte 1-16

Der zugeh?rige Abbildungsalgorithmus bietet nun zwei Optionen: uSAV und cSAV. uSAV ist f?r den laufenden Betrieb, also die Abbildung bestehender VNRs sowie der Abbildung neu hinzugekommener VNRs zust?dig und bildet diese zeitsparend ab. Das hei? die optimale Abbildung wird hier zwar selten erreicht, daf?r wird unter Betrachtung der Kosten-/Nutzen-Berechung ein g?nstiges Ergebnis erzielt. cSAV arbeitet, im Gegenzug zu uSAV, auf optimale Abbildungen hin. Um dies zu erreichen trennt cSAV Knoten- und Link-Abbildungen strikt, was zu enormen Verschlechterungen bei "no-path-splitting"-Anforderungen f?hrt.
Beide Algorithmen arbeiten mit den selben  Heuristiken und Sub-Algorithmen: Node Mapping und Link Mapping


\begin{center}
uSAV\newline
	\includegraphics[width=0.8\textwidth]{usav.pdf}\newline
\end{center}
\newpage
\begin{center}
cSAV\newline
	\includegraphics[width=0.8\textwidth]{csav.pdf}\newline
\end{center}
\newpage
\begin{center}
Knoten-Abbildungen\newline
	\includegraphics[width=0.8\textwidth]{nodemapping.pdf}\newline
\end{center}
\newpage
\begin{center}
Link-Abbildungen\newline
	\includegraphics[width=0.8\textwidth]{linkmapping.pdf}\newline
\end{center}



\subsection{Vergleich}
\label{subsec:svne_vergleich}



\section{Ungel?ste Probleme}
\label{sec:offenefragen}

\section{Schlussfolgerung und Ausblick}
\label{sec:schluss}
Frage: Auf welcher Ebene wird virtualisiert? Auf IP-Ebene? Was ist dann aber mit IP-Support-Protokollen (ARP)? Oder nicht-IP-Protokollen? Will ich ein Netzwerk virtualisieren, oder nur den IP-Verkehr? Verkapselung f?hrt zu Leistungseinbu?n. \cite{cabuk2007towards}



\bibliography{literatur}{}

\end{document}



