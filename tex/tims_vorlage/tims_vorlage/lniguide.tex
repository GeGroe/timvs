\documentclass{lni}


\IfFileExists{latin1.sty}{\usepackage{latin1}}{\usepackage{isolatin1}}
\usepackage{graphicx}


% Diese LaTeX-Vorlage entspricht dem Layout der GI-Edition Lecture Notes in Informatics (LNI), siehe auch: http://www.gi-ev.de/service/publikationen/lni/

\author{Vorname Name\\\\Seminar: Trends in Mobilen und Verteilten Systemen \\Wintersemester 2013/2014\\\\Lehrstuhl f�r Mobile und Verteilte Systeme\\Institut f�r Informatik\\Ludwig-Maximilians-Universit�t M�nchen}
\title{Thema der Ausarbeitung}

\begin{document}
\maketitle

\begin{abstract}
Hier folgt eine kurze Zusammenfassung des Themas sowie der wichtigsten Erkenntnisse und Ergebnisse. L�nge: ca. 200 W�rter.\end{abstract}

\section{Einleitung}
Die Ausarbeitung zum Seminar soll dem Layout der \textit{GI-Edition Lecture Notes in Informatics (LNI)} entsprechen. Die verwendete Literatur wird in der beiliegenden Datei \verb+literatur.bib+ verwaltet. Eine Referenz kann mittels des \verb+\cite{}+--Kommandos eingef�gt werden, z.B. \cite{Rivest78}.

\subsection{Hinweise f�r Abbildungen}
Abbildungen m�ssen als \texttt{.pdf}, \texttt{.png}, oder \texttt{.jpg} eingebunden werden. Beispiel:
\begin{figure}[htb]
  \begin{center}
    \includegraphics[width=1cm]{gilogo.pdf}
    \caption{\label{logo}Das Logo der GI}
  \end{center}
\end{figure}





\section{Sicherheitsaspekte bei virtuellen Netzwerk-Infrastrukturen}

\subsection{Herk�mmliche Gefahren}

\subsection{Spezielle Gefahren bei virtualisierten Umgebungen}

\subsection{VNE-Relevante Gefahren}


\section{Vermeidung von Gefahren via Secure VNE}

\subsection{Algorithmus 1}
\subsection{Algorithmus 2}
\subsection{Vergleich}

\section{Ungel�ste Probleme}


\section{Schlussfolgerung und Ausblick}


\bibliography{literatur}

\end{document}



