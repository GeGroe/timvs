\documentclass{lni}


\IfFileExists{latin1.sty}{\usepackage{latin1}}{\usepackage{isolatin1}}
\usepackage{graphicx}


% Diese LaTeX-Vorlage entspricht dem Layout der GI-Edition Lecture Notes in Informatics (LNI), siehe auch: http://www.gi-ev.de/service/publikationen/lni/

\author{Miran Mizani, Gerhard Gr�schl\\\\Seminar: Trends in Mobilen und Verteilten Systemen \\Wintersemester 2016/2017\\\\Lehrstuhl f�r Mobile und Verteilte Systeme\\Institut f�r Informatik\\Ludwig-Maximilians-Universit�t M�nchen}
\title{Sicherheitsaspekte bei Deployment virtueller Netzwerkinfrastrukturen}

\begin{document}
\maketitle

\begin{abstract}
Durch die �u�erst effiziente Nutzung von Hardware mittels Virtualisierung, steigt die Nachfrage nach virtualisierten Infrastrukturen enorm. Serviceprovider m�ssen die Hardwarebasis f�r ihre Dienste nicht mehr selbst unterhalten und geben ihre dahingehende Verantwortung an Infrastrukturanbieter weiter. Um die Sicherheit dieser Strukturen nicht zu vernachl�ssigen, arbeiten viele Forscher in diesem Bereich und versuchen effiziente Algorithmen mit integrierter Beachtung der Sicherheitsaspekte zu finden. Diese Arbeit soll einen �berblick und eine Klassifizierung der Gefahren die solche Konstrukte betreffen, sowie eine Analyse zweier unterschiedlicher Ans�tze zur Vermeidung m�glichst vieler Risiken zum Zeitpunkt der Planung �bermitteln.
\newline
Hier folgt eine kurze Zusammenfassung des Themas sowie der wichtigsten Erkenntnisse und Ergebnisse. L�nge: ca. 200 W�rter.\end{abstract}

\section{Einleitung}
Die Ausarbeitung zum Seminar soll dem Layout der \textit{GI-Edition Lecture Notes in Informatics (LNI)} entsprechen. Die verwendete Literatur wird in der beiliegenden Datei \verb+literatur.bib+ verwaltet. Eine Referenz kann mittels des \verb+\cite{}+--Kommandos eingef�gt werden, z.B.

\subsection{Hinweise f�r Abbildungen}
Abbildungen m�ssen als \texttt{.pdf}, \texttt{.png}, oder \texttt{.jpg} eingebunden werden. Beispiel:
\begin{figure}[htb]
  \begin{center}
    \includegraphics[width=1cm]{gilogo.pdf}
    \caption{\label{logo}Das Logo der GI}
  \end{center}
\end{figure}





\section{Sicherheitsaspekte bei virtuellen Netzwerk-Infrastrukturen}

\subsection{Herk�mmliche Gefahren}

\subsection{Spezielle Gefahren bei virtualisierten Umgebungen}

\subsection{VNE-Relevante Gefahren}


\section{Vermeidung von Gefahren via Secure VNE}
Um den Anforderungen der heutigen Gefahren zu gen�gen, ist es unumg�nglich s�mtliche Grunds�tze der IT-Sicherheit m�glichst fr�h in die Planung der gew�nschten Infrastruktur miteinzubeziehen. Nicht nur, weil das nachtr�gliche Schlie�en von Sicherheitsl�cken und Hinzuf�gen von Sicherheitskomponenten in finanzieller und zeitlicher Hinsicht 10 mal so teuer ist, wie die initiale Beachtung dieser Aspekte, sondern weil die vollst�ndige Sicherheit eines nachger�steten Systems kaum gew�hrleistet werden kann \cite{Cole}. "`Security-by-Design"' ist einer der wichtigsten Begriffe bei der Planung �ffentlich zug�nglicher Strukturen. Dementsprechend gro� ist die Nachfrage nach VNE-Algorithmen die bereits beim Prozess des Mappings m�glichst viele Sicherheitsaspekte beachten und abdecken. Die vorausgehende Klassifizierung der bekannten Gefahren in VNE-relevant und nicht-VNE-relevant bildet die Grundlage f�r unsere weiteren Untersuchungen. Die folgenden zwei Algorithmen wurden ausgew�hlt, um deren Beachtung und Vermeidung der VNE-relevanten Gefahren zu untersuchen und zu beurteilen.


\subsection{Algorithmus 1}
\subsection{Algorithmus 2}
\subsection{Vergleich}

\section{Ungel�ste Probleme}


\section{Schlussfolgerung und Ausblick}


\bibliography{literatur}{}

\end{document}



